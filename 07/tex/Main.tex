\documentclass[10pt, headings=normal, a4paper, bibliography=totoc]{scrartcl}
\widowpenalty = 10000
\clubpenalty = 10000

\parskip 12pt plus 1pt minus 1pt
\parindent 0pt

% Deutsch
\usepackage[ngerman]{babel}
\usepackage[utf8]{inputenc}
\usepackage[T1]{fontenc}
\usepackage{lmodern}
\usepackage{amsmath}
\usepackage{lscape}

\renewcommand\thesubsection{\alph{subsection})}

% Colors
\usepackage{color}
\definecolor{Gray}{gray}{0.9}

% Tables
\usepackage{colortbl}
\usepackage{tabularx}
\newcolumntype{A}{>{\centering\arraybackslash}X}
\newcolumntype{C}[1]{>{\setlength{\hsize}{#1\hsize}\addtolength{\hsize}{#1\tabcolsep}\centering\arraybackslash}X}

% Movies
\usepackage{movie15}

% \ding{}
\usepackage{pifont}
\usepackage{amssymb}

% exakte Bildposition
\usepackage{float}

% for comments
\usepackage{comment}

% BibTex
\usepackage{cite}

% Bilder
\usepackage{graphicx}
\usepackage[hang]{subfigure}
\setlength{\subfigcapmargin}{0.2cm}

% Pretty refs
\usepackage{prettyref}
\newrefformat{sec}{\ref{#1}:\ \emph{\nameref{#1}}}
\newrefformat{fig}{Abbildung \ref{#1}}
\newrefformat{tab}{Tabelle \ref{#1}}

% links
\usepackage{hyperref}
\usepackage{color}
\definecolor{links}{rgb}{0.2, 0.2, 0.3}
\hypersetup{
    colorlinks,
    citecolor=black,
    filecolor=black,
    linkcolor=black,
    urlcolor=black
}

% Kopf- und Fußzeile
\usepackage{scrpage2}
\pagestyle{scrheadings}
\ihead{\footnotesize{Felix Lauer (90404), Simon Schneegans (90405)}}
\ohead{\pagemark}
\chead{}
\cfoot{}
\setheadsepline{0.5pt}

% Formatierung
\usepackage{setspace}
\usepackage[a4paper, left=3cm, right=2.5cm, bottom=3cm, top=3cm]{geometry}


\renewcommand\thesubsection{\alph{subsection})}

\begin{document}

\thispagestyle{empty}
\ \\[3cm]
\begin{tabular}[b]{l}
    \Huge \textbf{Cryptographic Hash Functions} \hspace{12.5cm} \\[1mm]
    \hline
\end{tabular}

\begin{flushright}
    {\large Problem Session 3} \\[8cm]
\end{flushright}

\begin{tabular}[b]{l}
    \LARGE \textbf{Indifferentiability and Structural Weakness} \\[4cm]
\end{tabular}

\begin{tabular}{lcl}
    \textbf{Simon Schneegans} &\hspace{1cm} & \textbf{Felix Lauer} \\
    Computer Science and Media & \hspace{1cm}  & Computer Science and Media\\
    90405  &\hspace{1cm} & 90404 \\[1.5cm]
\end{tabular}

\newpage



\section{Entropy}

\subsection*{Preliminaries}

The entropy of a source $Q$ can be measured in different ways. For this task. we consider the \textbf{Shannon} and the \textbf{Min} entropy, which are defined as follows:

\begin{tabular}{ll}
  \textbf{Shannon Entropy:} & $H_2(x)= -\displaystyle\sum_{i=0}^{n-1}P(x_i)\cdot log_2 P(x_i)$ \\
  \textbf{Min Entropy:} & $H_{\infty}(x)= -log_2\ sup_{i=0...n-1}\ P(x_i)$
\end{tabular}

\subsection{Uniform distribution}

Since the source $Q$ follows a uniform distribution, the probability of one element $x_i$ of $Q$ can be written as

$$
P(x_i) = \frac{1}{62}, i \in \{0...61\}
$$

Accordingly, the source's Shannon entropy can be computed by

$$
H_2(x) = -\displaystyle\sum_{i=0}^{61}\frac{1}{62}\cdot log_2 P(\frac{1}{62})
\approx \text{\textbf{5.95}}
$$

and the Min Entropy can be calculated by
$$
H_{\infty}(x)= -log_2\ \frac{1}{62}
\approx \text{\textbf{5.95}}
$$

\subsection{High lower case probability}

Since the probability of a lower case letter in the source $Q$ is twice as high as the probability of an upper case letter, the probability of an element $x_i$'of $Q$ can be written as

$$
P(x_i) = \left\{
  \begin{array}{ll}
    \frac{1}{44} & : i \in \{0...25\}\\\\
    \frac{1}{88} & : i \in \{26...61\}
  \end{array}
\right.
$$

Accordingly, the source's Shannon entropy can be computed by

$$
H_2(x) = -\displaystyle\sum_{i=0}^{25}\frac{1}{44}\cdot log_2 P(\frac{1}{44}) + \displaystyle\sum_{i=26}^{61}\frac{1}{88}\cdot log_2 P(\frac{1}{88})
\approx \text{\textbf{5.87}}
$$

and the Min Entropy can be calculated by
$$
H_{\infty}(x)= -log_2\ \frac{1}{44}
\approx \text{\textbf{5.46}}
$$

\subsection{High digit probability}

Since the probability of a digit in the source $Q$ is given by $0.7$, the probability of an element $x_i$ of $Q$ can be written as

$$
P(x_i) = \left\{
  \begin{array}{ll}
    \frac{3}{520} & : i \in \{0...51\}\\\\
    \frac{7}{100} & : i \in \{52...61\}
  \end{array}
\right.
$$

Accordingly, the source's Shannon entropy can be computed by

$$
H_2(x) = -\displaystyle\sum_{i=0}^{51}\frac{3}{520}\cdot log_2 P(\frac{3}{520}) + \displaystyle\sum_{i=26}^{61}\frac{7}{100}\cdot log_2 P(\frac{7}{100})
\approx \text{\textbf{4.92}}
$$

and the Min Entropy can be calculated by
$$
H_{\infty}(x)= -log_2\ \frac{7}{100}
\approx \text{\textbf{3.84}}
$$

\section{Variant of HMAC}
\subsection{Forgery for HMAC* where the outer invocation of $H$ is invertible (i.e. $H'$)}

In this case, the key $K$ can be easily computed by
$$
K= H'^{-1}(HMAC)_{[0...|const_2|]} \oplus const_2
$$

where $H'^{-1}(X)$ is the inverse of $H'$, $HMAC$ is one round value of $HMAC^*$, $|const_2|$ is the bit length of $const_2$ and $X_{[0...|const_2|]}$ determines the first $|const_2|$ bits of $X$.

To get $U$, one has to guess with $O(2^{|U|})$.

This is an efficient forgery, since a part of the input of $H'$ is computable and the other part can be guessed with low effort, since $U$ is typically short.

\subsection{Forgery for HMAC* where the inner invocation of $H$ is invertible (i.e. $H'$)}

For that case, no efficient forgery can be found since no part of the input is efficiently computable.

\section{Weakness of ROMix}

As a consequence of the implementation of $ROMix$, the order of access to the different $V_j$ depends on the password. Hence, a spy process can trace which $V_j$ are accessed at the beginning of $ROMix$. While executing a brute-force attack, the adversary can check whether the current password leads to the same accesses and otherwise discard it.


\section{Programming Task (Python)}

The solution of this task is contained within the attached file \texttt{compute\_entro.py} and the output of the contained python program.


\end{document}
