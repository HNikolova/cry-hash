\documentclass[10pt, headings=normal, a4paper, bibliography=totoc]{scrartcl}
\widowpenalty = 10000
\clubpenalty = 10000

\parskip 12pt plus 1pt minus 1pt
\parindent 0pt

% Deutsch
\usepackage[ngerman]{babel}
\usepackage[utf8]{inputenc}
\usepackage[T1]{fontenc}
\usepackage{lmodern}
\usepackage{amsmath}
\usepackage{lscape}

\renewcommand\thesubsection{\alph{subsection})}

% Colors
\usepackage{color}
\definecolor{Gray}{gray}{0.9}

% Tables
\usepackage{colortbl}
\usepackage{tabularx}
\newcolumntype{A}{>{\centering\arraybackslash}X}
\newcolumntype{C}[1]{>{\setlength{\hsize}{#1\hsize}\addtolength{\hsize}{#1\tabcolsep}\centering\arraybackslash}X}

% Movies
\usepackage{movie15}

% \ding{}
\usepackage{pifont}
\usepackage{amssymb}

% exakte Bildposition
\usepackage{float}

% for comments
\usepackage{comment}

% BibTex
\usepackage{cite}

% Bilder
\usepackage{graphicx}
\usepackage[hang]{subfigure}
\setlength{\subfigcapmargin}{0.2cm}

% Pretty refs
\usepackage{prettyref}
\newrefformat{sec}{\ref{#1}:\ \emph{\nameref{#1}}}
\newrefformat{fig}{Abbildung \ref{#1}}
\newrefformat{tab}{Tabelle \ref{#1}}

% links
\usepackage{hyperref}
\usepackage{color}
\definecolor{links}{rgb}{0.2, 0.2, 0.3}
\hypersetup{
    colorlinks,
    citecolor=black,
    filecolor=black,
    linkcolor=black,
    urlcolor=black
}

% Kopf- und Fußzeile
\usepackage{scrpage2}
\pagestyle{scrheadings}
\ihead{\footnotesize{Felix Lauer (90404), Simon Schneegans (90405)}}
\ohead{\pagemark}
\chead{}
\cfoot{}
\setheadsepline{0.5pt}

% Formatierung
\usepackage{setspace}
\usepackage[a4paper, left=3cm, right=2.5cm, bottom=3cm, top=3cm]{geometry}


\begin{document}

\thispagestyle{empty}
\ \\[3cm]
\begin{tabular}[b]{l}
    \Huge \textbf{Cryptographic Hash Functions} \hspace{12.5cm} \\[1mm]
    \hline
\end{tabular}

\begin{flushright}
    {\large Problem Session 3} \\[8cm]
\end{flushright}

\begin{tabular}[b]{l}
    \LARGE \textbf{Indifferentiability and Structural Weakness} \\[4cm]
\end{tabular}

\begin{tabular}{lcl}
    \textbf{Simon Schneegans} &\hspace{1cm} & \textbf{Felix Lauer} \\
    Computer Science and Media & \hspace{1cm}  & Computer Science and Media\\
    90405  &\hspace{1cm} & 90404 \\[1.5cm]
\end{tabular}

\newpage



\section{Indifferentiability}

\textbf{Idea --} Exploit the knowledge about the public invertable permutation $ h $ to design/strengthen the oracle's simulator.

\textbf{Solution --} Design $ F $ as follows:

$$
F(z) = H(h^{-1}(z))
$$

When this construction is tested for indifferentiability in the ROM:

\begin{enumerate}
\item Ask for $ H(x) = y $.
\item Compute $ h(x) = z $.
\item Ask for $ F(z) $. Since $ F $ uses the inverse of $ h $ and passes the result to $ H $, $ F(z) $ will return $ y $ which renders the function indifferentiable by the terms of ROM.
\end{enumerate}

\hfill$\Box$

\section{Structural Weakness}

\textbf{Idea --} As $ H $ is an iterated COWHF, indermediate values based on chunks of its input are computed and used as initial values for subsequent calls of an internally used hash/compression function. This knowledge can be exploited since the data $ M $ can be considered constant.

\textbf{Solution --} Let the server behave as follows:

\begin{enumerate}
\item The server gets the data $ M $, computes the value $ h_{M} = H(M) $ and stores $ h_{M} $. Since $ H $ is iterated, $ H(M) $ is constant input for later iterations.
\item The server may now manipulate the data $ M $.
\item The client computes $ h_{i} = H(M\ ||\ C_{i}) $ with the secret challenge $ C_{i} $ and sends $ C_{i} $ to the server to check for integrity.
\item The server computes $ h'_{i} = H(C_{i}) $ while using the stored $ h_{M} $ as initial value for the first iteration in $ H $. As a consequence, $ h'_{i} $ equals $ h_{i} $.
\end{enumerate}

Since the input for the iterated $ H $ is $ M\ ||\ C_{i} $, the concatenation of $ C_{i} $ has now influence on internal iterations on all chunks of $ M $. This allows the server to pre-compute $ H(M) $ and manipulate the data afterwards.

\newpage
\section{Greedy Adversary}

\textbf{Idea --} Use the principle of multi-collisions as in the Joux-attack. If we can efficiently find 20 collisions for an iterated structure, we can search for one preimage for this structure and get $ 2^{20} $ preimages through permutation.

\textbf{Solution --} Set up an iterated structure as follows:

\begin{enumerate}
\item Use $ k = 20 $ iterations of $ H $.
\item For each iteration find one collision $ x'_{i} $ for $ x_{i} $ with $ n = |x_{i}| = 128 $ and $ i \in \{0...k\} $. As shown by the Joux-attack, the following equations hold:
$$
O(1\ coll) = 2^{\frac{n}{2}}
$$ thus
$$
O(20\ coll) = 2^{\frac{n}{2}} \cdot 20
$$
Since $ n = 128 $ :
$$
O(20\ coll) = 2^{64} \cdot 20
$$
Because $ 20 < 2^{5} $, we can substitute $ 20 $ and get
$$
O(20\ coll) < 2^{69}
$$
thus finding 32 collisions has an efficiency of $ O(2^{69}) $
\item Find one $ x_{k+1} $ such that $ H(x_{0} || x_{1} || ... || x_{k+1} ) = Y $ . Since the output $ Y $ is truncated, we can find $ x_{k+1} $ with $ O(2^{56}) $.
\item By permutation of each $ x_{i} $ and $ x'_{i} $, we will find $ 2^{20} $ preimages for $ Y $.

\end{enumerate}

The overall consumption is determined by the structural setup ($ O(2^{69}) $) and finding a preimage ($ O( 2^{56} ) $). This yields $ O(2^{69} + 2^{56}) \sim O(2^{69} ) $.

\section{Programming Task}

The code is attached to the mail! Please have a look at the file \texttt{collisions.py}.

Regarding to the gummy bears: We found multiple 8-byte collisions.

\end{document}
