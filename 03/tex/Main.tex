\documentclass[10pt, headings=normal, a4paper, bibliography=totoc]{scrartcl}
\widowpenalty = 10000
\clubpenalty = 10000

\parskip 12pt plus 1pt minus 1pt
\parindent 0pt

% Deutsch
\usepackage[ngerman]{babel}
\usepackage[utf8]{inputenc}
\usepackage[T1]{fontenc}
\usepackage{lmodern}
\usepackage{amsmath}
\usepackage{lscape}

\renewcommand\thesubsection{\alph{subsection})}

% Colors
\usepackage{color}
\definecolor{Gray}{gray}{0.9}

% Tables
\usepackage{colortbl}
\usepackage{tabularx}
\newcolumntype{A}{>{\centering\arraybackslash}X}
\newcolumntype{C}[1]{>{\setlength{\hsize}{#1\hsize}\addtolength{\hsize}{#1\tabcolsep}\centering\arraybackslash}X}

% Movies
\usepackage{movie15}

% \ding{}
\usepackage{pifont}
\usepackage{amssymb}

% exakte Bildposition
\usepackage{float}

% for comments
\usepackage{comment}

% BibTex
\usepackage{cite}

% Bilder
\usepackage{graphicx}
\usepackage[hang]{subfigure}
\setlength{\subfigcapmargin}{0.2cm}

% Pretty refs
\usepackage{prettyref}
\newrefformat{sec}{\ref{#1}:\ \emph{\nameref{#1}}}
\newrefformat{fig}{Abbildung \ref{#1}}
\newrefformat{tab}{Tabelle \ref{#1}}

% links
\usepackage{hyperref}
\usepackage{color}
\definecolor{links}{rgb}{0.2, 0.2, 0.3}
\hypersetup{
    colorlinks,
    citecolor=black,
    filecolor=black,
    linkcolor=black,
    urlcolor=black
}

% Kopf- und Fußzeile
\usepackage{scrpage2}
\pagestyle{scrheadings}
\ihead{\footnotesize{Felix Lauer (90404), Simon Schneegans (90405)}}
\ohead{\pagemark}
\chead{}
\cfoot{}
\setheadsepline{0.5pt}

% Formatierung
\usepackage{setspace}
\usepackage[a4paper, left=3cm, right=2.5cm, bottom=3cm, top=3cm]{geometry}


\begin{document}

\thispagestyle{empty}
\ \\[3cm]
\begin{tabular}[b]{l}
    \Huge \textbf{Cryptographic Hash Functions} \hspace{12.5cm} \\[1mm]
    \hline
\end{tabular}

\begin{flushright}
    {\large Problem Session 3} \\[8cm]
\end{flushright}

\begin{tabular}[b]{l}
    \LARGE \textbf{Indifferentiability and Structural Weakness} \\[4cm]
\end{tabular}

\begin{tabular}{lcl}
    \textbf{Simon Schneegans} &\hspace{1cm} & \textbf{Felix Lauer} \\
    Computer Science and Media & \hspace{1cm}  & Computer Science and Media\\
    90405  &\hspace{1cm} & 90404 \\[1.5cm]
\end{tabular}

\newpage



\setcounter{section}{1}
\section{Similarity Measures}

\textbf{(d)}

Possibility 1: 
$$
d = \frac{1}{\rho(x_1,x_2)} - 1
$$

Possibility 2: 
$$
d = \frac{1-\rho}{\rho}
$$

\textbf{(e)} The euclidean distance $ d $:

$$
d(x_1, x_2) = \sqrt{\sum_{i=0}^{n}(x_{1i} - x_{2i})^2}
$$

The cosine similarity $ \rho $ for unit vectors:

$$
\rho(x_1, x_2) = \sum_{i=0}^{n}(x_{1i} * x_{2i})
$$

The length of $ x_1 $ and $ x_2 $ equals $ 1 $:

$$
\sqrt{\sum_{i=0}^{n}x_{1i}^2} = \sqrt{\sum_{i=1}^{n}x_{2i}^2} = \sum_{i=1}^{n}x_{1i}^2 = \sum_{i=1}^{n}x_{2i}^2 = 1
$$

Some equation reformation:

$$
\begin{aligned}
d(x_1, x_2)^2 &= \sum_{i=0}^{n}x_{1i}^2 - 2 * \sum_{i=0}^{n}x_{1i}x_{2i} + \sum_{i=0}^{n}x_{2i}^2 \\
 &= 1 - 2\rho(x_1, x_2) + 1\\
 &= 2 * (1-\rho(x_1, x_2))\\
\end{aligned}
$$

\pagebreak
\textbf{(f)} We assume that the genetic makeup is stored as follows:

$$
\begin{array}{ccc}
     &   x_1 & x_2\\
    g_1 & 1 &  0\\
    g_2 & 0 &  1\\
    g_3 & 0 &  0\\
    . & &\\
    . & &\\
    . & &\\
    g_n & 1 &  1 \\
\end{array}
$$
    
Where the $g_i$ are all possible genes of the species and 1 and 0 indicate if 
the $g_i$ is present in the vectors $x_1$ and $x_2$. 
Given that, we would choose the Jaccard or Simple Matching Coefficient because 
they show the percentual similarity of the vectors. 
The Hamming distance would be rendered useless, because it only shows the 
difference in the total amount of genes between the two individuals.  

\setcounter{section}{5}
\section{\emph{k}-Means}

\textbf{(a)} Each time an object is selected it's class should not have been selected before:

$$
p(x) = \frac{k}k * \frac{k-1}k * ... * \frac 1 k = \prod_{i=0}^{k-1} \frac{k-i}k = \frac{k!}{k^k}
$$

\textbf{(b)}
\begin{figure}[H]
\includegraphics[scale=0.75]{pictures/means.png}
\end{figure}

\section{Cluster Analysis Principles}

Which of the following statements are true?

$\Box$ \emph{k}-means is a supervised algorithm since the centroids are specified.

$\Box$ The runtime of \emph{k}-medoid is higher than that of k-means due to the medoid computation.

$\boxtimes$ Density-based cluster analysis is more efficient than single link.

$\Box$ DBSCAN is particularly efficient in high dimensions.

\end{document}
