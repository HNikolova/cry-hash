\documentclass[10pt, headings=normal, a4paper, bibliography=totoc]{scrartcl}
\widowpenalty = 10000
\clubpenalty = 10000

\parskip 12pt plus 1pt minus 1pt
\parindent 0pt

% Deutsch
\usepackage[ngerman]{babel}
\usepackage[utf8]{inputenc}
\usepackage[T1]{fontenc}
\usepackage{lmodern}
\usepackage{amsmath}
\usepackage{lscape}

\renewcommand\thesubsection{\alph{subsection})}

% Colors
\usepackage{color}
\definecolor{Gray}{gray}{0.9}

% Tables
\usepackage{colortbl}
\usepackage{tabularx}
\newcolumntype{A}{>{\centering\arraybackslash}X}
\newcolumntype{C}[1]{>{\setlength{\hsize}{#1\hsize}\addtolength{\hsize}{#1\tabcolsep}\centering\arraybackslash}X}

% Movies
\usepackage{movie15}

% \ding{}
\usepackage{pifont}
\usepackage{amssymb}

% exakte Bildposition
\usepackage{float}

% for comments
\usepackage{comment}

% BibTex
\usepackage{cite}

% Bilder
\usepackage{graphicx}
\usepackage[hang]{subfigure}
\setlength{\subfigcapmargin}{0.2cm}

% Pretty refs
\usepackage{prettyref}
\newrefformat{sec}{\ref{#1}:\ \emph{\nameref{#1}}}
\newrefformat{fig}{Abbildung \ref{#1}}
\newrefformat{tab}{Tabelle \ref{#1}}

% links
\usepackage{hyperref}
\usepackage{color}
\definecolor{links}{rgb}{0.2, 0.2, 0.3}
\hypersetup{
    colorlinks,
    citecolor=black,
    filecolor=black,
    linkcolor=black,
    urlcolor=black
}

% Kopf- und Fußzeile
\usepackage{scrpage2}
\pagestyle{scrheadings}
\ihead{\footnotesize{Felix Lauer (90404), Simon Schneegans (90405)}}
\ohead{\pagemark}
\chead{}
\cfoot{}
\setheadsepline{0.5pt}

% Formatierung
\usepackage{setspace}
\usepackage[a4paper, left=3cm, right=2.5cm, bottom=3cm, top=3cm]{geometry}


\renewcommand\thesubsection{\alph{subsection})}

\begin{document}

\thispagestyle{empty}
\ \\[3cm]
\begin{tabular}[b]{l}
    \Huge \textbf{Cryptographic Hash Functions} \hspace{12.5cm} \\[1mm]
    \hline
\end{tabular}

\begin{flushright}
    {\large Problem Session 3} \\[8cm]
\end{flushright}

\begin{tabular}[b]{l}
    \LARGE \textbf{Indifferentiability and Structural Weakness} \\[4cm]
\end{tabular}

\begin{tabular}{lcl}
    \textbf{Simon Schneegans} &\hspace{1cm} & \textbf{Felix Lauer} \\
    Computer Science and Media & \hspace{1cm}  & Computer Science and Media\\
    90405  &\hspace{1cm} & 90404 \\[1.5cm]
\end{tabular}

\newpage



\section{Differential Cryptanalysis of Variants of CubeHash}

\subsection{CubeHash*}

We introduce a MSB-difference in the $1^{st}$ word of the internal state ($w_1$).

\begin{itemize}
    \item First $\boxplus$ will propagate the difference with propability $Pr = 1$.
    \item First $swap$ won't do anything.
    \item First $\oplus$ will propagate the difference to the word $w_{16}$ with propability $Pr = 1$.
    \item Second $swap$ will move the difference to the MSB of $w_3$ with propability $Pr = 1$.
    \item Second $\boxplus$ will propagate the differences $w_1$ and $w_3$ with propability $Pr = 1$.
    \item Third $swap$ will move the difference to the MSB of $w_21$ with propability $Pr = 1$.
    \item Second $\oplus$ will propagate the difference to the words $w_{17}$, $w_{19}$ and $w_{21}$ with propability $Pr = 1$.
    \item Fourth $swap$ will move the difference to the MSB of $w_4$ with propability $Pr = 1$.
\end{itemize}

Finally, with propability $Pr = 1$, the MSB of $w_4$, $w_{17}$, $w_{19}$ and $w_{21}$ will be flipped. If not flipping the MSB, each $\boxplus$ will introduce a $Pr = 0.5$ for each flipped bit in their input. This would lead to a final propability of $Pr = 0.125$. Obviously, the security is significantly decreased regarding to differential cryptanalysis. Regarding to symmetric states: They are still possible.

\subsection{CubeHash**}

We introduce a MSB-difference in the $i^{th}$ word of the first 16 state words of the internal state ($w_i$).

\begin{itemize}
    \item First $\boxplus$ will propagate the difference with propability $Pr = 1$.
    \item First $\lll$ won't do anything.
    \item First $\oplus$ will propagate the difference to the word $w_{i+16}$ with propability $Pr = 1$.
    \item Second $\boxplus$ will eliminate difference in $w_i$ with propability $Pr = 1$.
    \item Second $\lll$ moves difference to $11^{th}$ position with propability $Pr = 1$.
    \item Second $\oplus$ won't do anything.
\end{itemize}

Finally, with propability $Pr = 1$, the $11^{th}$ bit of $w_{i+16}$ will be flipped. If not flipping the MSB, each $\boxplus$ will introduce a $Pr = 0.5$ for each flipped bit in their input. This would lead to a final propability of $Pr = 0.25$. Obviously, the security is significantly decreased regarding to differential cryptanalysis. Regarding to symmetric states: They are still possible.


\section{Differential Cryptanalysis of BLAKE}

\subsection{}

\begin{itemize}
    \item Introduce MSB-differences in the following input variables: $a$, $M_0$ and $M_3$.
    \item After the first round, there is no difference left with propability $Pr = 1$.
    \item The second and third round will introduce no new differences.
    \item The last round introduces the following differences:
    \begin{itemize}
        \item $a'$: MSB with $Pr = 1$
        \item $b'$: $15^{th}$ bit with $Pr = 1$
        \item $c'$: $8^{th}$ bit with $Pr = 0.5$
        \item $d'$: $8^{th}$ bit with $Pr = 1$
    \end{itemize}
    \item This leads to a total propability of 0.5
\end{itemize}

\subsection{}

\begin{itemize}
    \item Introduce MSB-differences in the following input variables: $M_2$ and $M_3$.
    \item The first differences are introduced in the third round. They lead to differences in the four intermediate chaining values:
    \begin{itemize}
        \item $a'$: MSB with $Pr = 1$
        \item $b'$: $28^{th}$ bit with $Pr = 1$
        \item $c'$: $16^{th}$ bit with $Pr = 0.5$
        \item $d'$: $16^{th}$ bit with $Pr = 1$
    \end{itemize}
    \item In the last round these differences are propagated through the structure, leading to the following flipped bits in the final chaining values:
    \begin{itemize}
        \item $a''$: $28^{th}$ bit with $Pr = 0.5$
        \item $b''$: $3^{th}$ bit, $11^{th}$ bit and $28^{th}$ bit with $Pr = 0.5$
        \item $c''$: $4^{th}$ bit, $16^{th}$ bit and $24^{th}$ bit with $Pr = 2^{-5}$
        \item $d''$: $4^{th}$ bit and $24^{th}$ bit with $Pr = 0.5$
    \end{itemize}
    \item This leads to a total propability of $Pr = 2^{-8}$
\end{itemize}

\section{Differential Cryptanalysis of Skein*}

\begin{itemize}
    \item Introduce MSB-differences in the $2^{nd}$ and $4^{th}$ block.
    \item After one round this leads to MSB-differences in the $1^{st}$ and $3^{rd}$ block.
    \item After second round there are MSB-differences in the $2^{nd}$ and $4^{th}$ block again.
    \item After 72 rounds there will be MSB-differences in the $2^{nd}$ and $4^{th}$ block with propability of $Pr = 1$.
\end{itemize}

\section{Programming Task (Python)}

\begin{table}[H]
    \begin{center}
        \begin{tabularx}{\textwidth}[]{XXXX}
            \hline
            \rowcolor{Gray} \textbf{Pobability} & \textbf{Path length} & \textbf{Rotation Constant} & \textbf{Bit position of input difference} \\
            \hline
            1  &  72  &  0  &  63 \\
            1  &  72  &  0  &  127 \\
            1  &  72  &  0  &  191 \\
            1  &  72  &  0  &  255 \\
            2.11758236814e-22  &  72  &  32  &  63 \\
            2.11758236814e-22  &  72  &  32  &  127 \\
            2.11758236814e-22  &  72  &  32  &  191 \\
            2.11758236814e-22  &  72  &  32  &  255 \\
            4.48415508584e-44  &  72  &  0  &  0 \\
            4.48415508584e-44  &  72  &  0  &  1 \\
            4.48415508584e-44  &  72  &  0  &  2 \\
            4.48415508584e-44  &  72  &  0  &  3 \\
            4.48415508584e-44  &  72  &  0  &  4 \\
            ... &&& \\
            1.59309191113e-58  &  72  &  16  &  95 \\
            1.59309191113e-58  &  72  &  48  &  95 \\
            1.59309191113e-58  &  72  &  16  &  223 \\
            1.59309191113e-58  &  72  &  48  &  223 \\
            3.88938454866e-62  &  72  &  16  &  31 \\
            3.88938454866e-62  &  72  &  48  &  31 \\
            3.88938454866e-62  &  72  &  16  &  159 \\
            3.88938454866e-62  &  72  &  48  &  159 \\
            9.49556774576e-66  &  72  &  32  &  0 \\
            9.49556774576e-66  &  72  &  32  &  1 \\
            9.49556774576e-66  &  72  &  32  &  2 \\
            9.49556774576e-66  &  72  &  32  &  3 \\
            9.49556774576e-66  &  72  &  32  &  4 \\
            ... &&& \\
            2.31825384418e-69  &  72  &  16  &  63 \\
            2.31825384418e-69  &  72  &  48  &  63 \\
            2.31825384418e-69  &  72  &  16  &  191 \\
            2.31825384418e-69  &  72  &  48  &  191 \\
            5.65979942427e-73  &  72  &  16  &  127 \\
            5.65979942427e-73  &  72  &  48  &  127 \\
            5.65979942427e-73  &  72  &  16  &  255 \\
            5.65979942427e-73  &  72  &  48  &  255 \\
            ... &&& \\
            \hline
        \end{tabularx}
    \end{center}
\end{table}


\end{document}
