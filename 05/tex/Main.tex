\documentclass[10pt, headings=normal, a4paper, bibliography=totoc]{scrartcl}
\widowpenalty = 10000
\clubpenalty = 10000

\parskip 12pt plus 1pt minus 1pt
\parindent 0pt

% Deutsch
\usepackage[ngerman]{babel}
\usepackage[utf8]{inputenc}
\usepackage[T1]{fontenc}
\usepackage{lmodern}
\usepackage{amsmath}
\usepackage{lscape}

\renewcommand\thesubsection{\alph{subsection})}

% Colors
\usepackage{color}
\definecolor{Gray}{gray}{0.9}

% Tables
\usepackage{colortbl}
\usepackage{tabularx}
\newcolumntype{A}{>{\centering\arraybackslash}X}
\newcolumntype{C}[1]{>{\setlength{\hsize}{#1\hsize}\addtolength{\hsize}{#1\tabcolsep}\centering\arraybackslash}X}

% Movies
\usepackage{movie15}

% \ding{}
\usepackage{pifont}
\usepackage{amssymb}

% exakte Bildposition
\usepackage{float}

% for comments
\usepackage{comment}

% BibTex
\usepackage{cite}

% Bilder
\usepackage{graphicx}
\usepackage[hang]{subfigure}
\setlength{\subfigcapmargin}{0.2cm}

% Pretty refs
\usepackage{prettyref}
\newrefformat{sec}{\ref{#1}:\ \emph{\nameref{#1}}}
\newrefformat{fig}{Abbildung \ref{#1}}
\newrefformat{tab}{Tabelle \ref{#1}}

% links
\usepackage{hyperref}
\usepackage{color}
\definecolor{links}{rgb}{0.2, 0.2, 0.3}
\hypersetup{
    colorlinks,
    citecolor=black,
    filecolor=black,
    linkcolor=black,
    urlcolor=black
}

% Kopf- und Fußzeile
\usepackage{scrpage2}
\pagestyle{scrheadings}
\ihead{\footnotesize{Felix Lauer (90404), Simon Schneegans (90405)}}
\ohead{\pagemark}
\chead{}
\cfoot{}
\setheadsepline{0.5pt}

% Formatierung
\usepackage{setspace}
\usepackage[a4paper, left=3cm, right=2.5cm, bottom=3cm, top=3cm]{geometry}


\renewcommand\thesubsection{\alph{subsection})}

\begin{document}

\thispagestyle{empty}
\ \\[3cm]
\begin{tabular}[b]{l}
    \Huge \textbf{Cryptographic Hash Functions} \hspace{12.5cm} \\[1mm]
    \hline
\end{tabular}

\begin{flushright}
    {\large Problem Session 3} \\[8cm]
\end{flushright}

\begin{tabular}[b]{l}
    \LARGE \textbf{Indifferentiability and Structural Weakness} \\[4cm]
\end{tabular}

\begin{tabular}{lcl}
    \textbf{Simon Schneegans} &\hspace{1cm} & \textbf{Felix Lauer} \\
    Computer Science and Media & \hspace{1cm}  & Computer Science and Media\\
    90405  &\hspace{1cm} & 90404 \\[1.5cm]
\end{tabular}

\newpage



\section{MD4 \textendash\ Differential Cryptanalysis I}

\subsection{}
We will describe the exact differential path in section \textbf{b)}. Since in the general case the most significant bit (MSB) is not flipped (as it is the case for task b), the 4th step may introduce a carry bit through calling \textit{Fun} as wall as through adding $X_1'$. Hence the overall probability of the differential path is $\frac{1}{64}$.

\subsection{}
We consider $X_0 \oplus X_0' = 2^{28}$ and $X_1 \oplus X_1' = 2^{31}$. The corresponding differential path looks as follows:
\begin{itemize}

\item Steps 3 to 16 do not introduce any difference.
\item Step 17:
\begin {itemize}
  \item Since the 3rd bit of $X_{0}'$ is flipped, the probability of no other bit being flipped through \textit{Fun} is $\frac{1}{2}$.
  \item Through the final rotation at the end of the step the flipped bit will become the MSB
\end{itemize}
\item Steps 18 to 20: Each step introduces a probability of $\frac{1}{2}$ for no new difference being generated.
\item Step 21: Since the first bit of $X_1'$ is flipped, the previously introduced difference is annihilated and the differential path is finished. Its overall probability is $\frac{1}{16}$.
\item Steps 22 to 32: No new difference is introduced.
\end{itemize}

\subsection{}
The probability for this characteristic will decrease because the additive difference may introduce more than one flipped bit, which then have to be eliminated.

\section{MD4 \textendash\ Differential Cryptanalysis II}
We set $X_i = X_i'$ for $i \notin \{2,3\}$; steps 5-36 and $i \in \{0,15\}$ and choose the differential characteristic to be $X_2 \oplus X_2' = 2^{28}$ and $X_3 \oplus X_3' = 2^{31}$. The differential path is built in analogy to task \textbf{1b)} whereas the first difference occures in step 25. The introduced flipped bit is annihilated in step 29 and the overall probability of this path is $\frac{1}{16}$.

\newpage

\section{MD4-NSA}
We set $X_i = X_i'$ for $i \notin \{1,2\}$; steps 4-48 and $i \in \{0,15\}$ and choose the differential characteristic to be $X_1 \oplus X_1' = 2^{15}$ and $X_2 \oplus X_2' = 2^{31}$. The corresponding differential path looks as follows:

\begin{itemize}
\item Steps 4 to 20 do not introduce any difference.
\item Step 21: $X_1'$ occures the first time. Since the 16th bit is flipped, a carry bit may be introduced. With probability $\frac{1}{2}$ this does not happen. Through rotation by 16 bits the flipped bit becomes the MSB.
\item Steps 22 to 24: Each step introduces a probability of $\frac{1}{2}$ for no new difference being generated through \textit{Fun}.
\item Step 25: $X_2'$ occures the first time. With probability of $\frac{1}{2}$ the difference introduced by the previous steps is annihilated. Accumulating the probabilities of the previous steps there is a probability of $\frac{1}{16}$ for no difference after the 25th step.
\item Steps 26 to 36: No new difference is introduced.
\item Step 37: $X_2'$ occures the second time. Since the MSB is flipped, no carry bit can be produced. The flipped bit is rotated to position 16.
\item Steps 38 to 40: Each step introduces a probability of $\frac{1}{2}$ for no new difference being generated through \textit{Fun}.
\item Step 41: $X_1'$ occures the second time. Because the 16th bit is flipped, the difference introduced by the previous steps may be annihilated. Since there is a chance of $\frac{1}{2}$ that no carry bit is introduced and a chance of $\frac{1}{2}$ for \textit{Fun} to not alter the 16th bit, the probability of no new difference being introduced is $\frac{1}{4}$.
\item Steps 42 to 48: No new difference is introduced.

\end{itemize}

The overall probability of this differential path is $\frac{1}{512}$. This path can be considered very long since it covers most of the steps (44 out of 48) of MD4-NSA. Since the constant rotational factor is exactly half of the message block length (32 bits), this knowledge can be exploited by flipping the 16th bit. As a result MD4-NSA is more vulnerable to differential cryptanalysis.

\section{Merkle Hash Tree}
We define $H$ to be the Merkle Hash Tree construction using the compression function $F$.

\textbf{To show:}

\begin{equation}
F\ \text{collision resitant} \Rightarrow H\ \text{collision resistant}
\end{equation}

We choose the indirect proof technique. Thus we are going to show

\begin{equation}
H\ \text{not collision resitant} \Rightarrow F\ \text{not collision resistant}
\end{equation}

\newpage

\textbf{Proof:}

\begin{itemize}
\item Let $M$ and $M'$ be a pair of messages with $M \neq M'$ and $H(M) = H(M')$, i.e. $M$ and $M'$ produce colliding hash values.
\item To prove (2) we need to find a part of the construction where a call to $F$ must collide.
\item We consider to cases: $|M| = |M'|$ and $|M| \neq |M'|$
\item Case 1 ($|M| \neq |M'|$): Since the last calls to $F$ respectively take $|M|$ and $|M'|$ as input and these lengths are different, there must be a collision in $F$, i.e. $F$ is not collision resistant.
\item Case 2 ($|M| = |M'|$):

\begin{itemize}
\item If the results of the last but one call to $F$ for $M$ and $M'$ are different, there must be a collision in the last call to $F$.
\item Else the last but one call to $F$ collided or got the same input for both $M$ and $M'$. If that is the case, repeat the previous argument until a collision for $F$ is found or the construction's leaves are reached.
\item If the leaves are reached, a collision is found for all pairs of $m_i$, $m_i'$ where $m_i \neq m_i'$
\end{itemize}

\end{itemize}
\hfill$\Box$

\end{document}
